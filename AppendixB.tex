
\newpage\
\chapter{Choosing weight for event average}

The double angular brakets in flow analysis means usually means that one for average over all particles, and the other for the average over all events.

	$$ \left\langle  \left\langle . . . \right\rangle  \right\rangle = \langle \langle . . .  \rangle _{particles} \rangle _{events} $$

And for the average over events, the finding proper event weights are important to get the results without bias. 

One approach to get event averaged results, is set the all event weights as 1 (equal weights). For example, measure the flow coefficient($v_n$) in given centrality, we measure $v_n$ for each events, and calculate mean $v_n$ as average of measured $v_n$ with uniform weights(or materially no weights). But this approach is not correct in case of multi-particle correlations. 

In this analysis of $SC(m,n)$, the measurement performed in two procedure, first over all distinct particle quadruplets in an event, and then in the second step the single-event averages were weighted with  "number of combinations" for correct event weight.

To prove this, let's think a set of N events, where the multiplicity of each event is $M_{i}$ where the $i$ denote for $i$th event. 

Then the equation for 2-particle correlation Eq.\ref{eq:2pcorr} can be written as like

\begin{equation}
	\langle \langle 2 \rangle \rangle  \equiv 	\frac{\sum_{i=1}^{N} \sum_{a,b=1}^{M_i} e^{in(\phi_{i,a}-\phi_{i,b})} }{\sum_{i=1}^{N} \sum_{a,b=1}^{M_i}}
	\label{eq:apendix2pcorr}
\end{equation}


with the constraints $a \neq b$.  The equation \ref{eq:apendix2pcorr} implies that, we calculated every possible pairs of particle in an events and by using constraints, we eliminated all contributions from self-correlations. If we take two distinct pairs of particles, one formed in event A and another formed in event B, then the above definition ensures that these two distinct pairs of particles will be taken into account at equal footing (i.e. a unit weight has being assigned to each distinct pair of particles in any event in Eq.\ref{eq:apendix2pcorr}). The denominator in definition simply constant the total number of all such distinct pairs in all events. In general form, for the definition of the $i$th event 

\begin{equation}
	 \langle 2 \rangle _{i}  \equiv 	\frac{ \sum_{a,b=1}^{M_i} e^{in(\phi_{i,a}-\phi_{i,b})} }{ \sum_{a,b=1}^{M_i}}
	\label{eq:appendix2pcorrith}
\end{equation}
\smallskip

And the total number the distinct pairs in the $i$th events will be evaluated as follows 

\begin{equation}
 \sum_{a,b=1}^{M_i} = \left( \sum_{a=1}^{M_i} \right) \left)\sum_{b=1}^{M_i} \right) - \sum_{a=b=1}^{M_i} = M_i^2 - M_i = M_i(M_i-1)
\end{equation}
\smallskip

Extend to all event sets,

\begin{equation}
	langle \langle 2 \rangle \rangle = \frac{\sum_{i=1}^{N} \sum_{a,b=1}^{M_i} e^{in(\phi_{i,a}-\phi_{i,b})} }{\sum_{i=1}^{N} M_i(M_i-1)}
\end{equation}

and 

\begin{equation}
	 \langle 2 \rangle _{i}  \equiv 	\frac{ \sum_{a,b=1}^{M_i} e^{in(\phi_{i,a}-\phi_{i,b})} }{ M_i(M_i-1)}
\end{equation}

Then, we immediately get the following results by insert above equation to the original Eq. \ref{eq:apendix2pcorr}


\begin{equation}
	\langle \langle 2 \rangle \rangle  = 	\frac{\sum_{i=1}^{N} M_i(M_i-1) \times \frac{\sum_{a,b=1}^{M_i} e^{in(\phi_{i,a}-\phi_{i,b})}}{M_i(M_i -1 )} }{\sum_{i=1}^{N} M_i(M_i - 1)}
	\label{eq:rhs}
\end{equation}

and finally we get, 

\begin{equation}
		\langle \langle 2 \rangle \rangle  = 	\frac{\sum_{i=1}^{N}M_i(M_i-1) \times \langle 2 \rangle _{i} }{\sum_{i=1}^{N}M_i(M_i-1)}
		\label{eq:weight_final}
\end{equation}

As seen in Eq. \ref{eq:weight_final}, the event weight number of combinations has to be used to weight single event average $\langle 2 \rangle$ to obtain exactly the all event average $\langle \langle 2 \rangle \rangle $. The prof of 4-particle correlation can be done in similar way. 



