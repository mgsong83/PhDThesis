% !TEX root = main_org.tex


\chapter{Conclusion and Outlook}

As the strong evidence of QGP, flow has been studied in detail during the past few decades. But only a few studies have been done about correlation between flow harmonics and leads to the following questions; How do $v_n$ and $\epsilon_n$ fluctuate and what is the underlying probability density function ($p.d.f$) of their distribution? How are the initial geometry fluctuations reflected in differential flow measurements? What is the relationship between different harmonic event plane angle? What is the relationship between the flow coefficient of different harmonics? The answers to above questions (especially for the last) \textit{Symmetric Cumulants} ($SC(m,n)$) have been introduced as the key observable to measure correlation between flow harmonic ``magnitudes" $v_m$ and $v_n$.



As the result of $SC(m,n)$, We have found that fluctuations of $v_2$-$v_3$ and $v_3$-$v_4$ are anti-correlated in all centralities while fluctuations of $v_2$-$v_4$, $v_2$-$v_5$ and $v_3$-$v_5$ are correlated for all centralities. The various hydrodynamic calculations and model simulations were studied together as a reference. The large differences between data and MC-Glauber studies confirmed that the correlation were not able to explained by only linear response of initial conditions, and there is certain non-linear contribution of initial fluctuations. The analysis with higher order flow harmonics provides that the correlation between higher order flow harmonics and lower order harmonics is likely due to the non-linear contributions. It  also indicates that the higher order flow can be understood as the superposition of the lower order flow harmonics. This analysis will help constrain the theoretical description of the fluid close the freeze-out temperature which is probably the least understood part of hydrodynamic calculation. \cite{Teaney:2012ke} \cite{Yan:2015dh}

 None of the existing models and hydrodynamic calculation with the different parameterizations of the temperature dependence of $\eta/s$ couldn't exactly capture the data quantitatively. Furthermore, the sign of $v_3$-$v_2$ correlation in most central collision range(0-10\%) was found to be different between the data and hydrodynamic model calculations.  In the most central collisions the anisotropies originate mainly from fluctuations, i.e. the initial ellipsoidal geometry characteristic for mid-central collisions plays little role in this regime.  Hence this observation might help to understand the details of the fluctuations in initial conditions.
 
 It is suggested that the $SC(m,n)$ is more sensitive to both initial conditions and hydrodynamic property $\eta/s$ than single flows.  In addition, we have found that the different order harmonic correlations have different sensitivities to the initial conditions and the system properties. Therefore they have discriminating power on separating the role of the $\eta/s$  from the initial conditions to the final state particle anisotropies.

The comparisons to VISH2+1 calculation show that all the models with large $\eta/s$ regardless of the initial conditions failed to capture the centrality dependence of higher order correlations, more clearly than lower order harmonic correlations. 
Based on the tested model parameters, the $\eta/s$ should be small and AMPT initial condition is favored by the data. A quite clear separation of the correlation strength between different initial conditions is observed for these higher order harmonic correlations compared to the lower order harmonic correlations.

We have found that $v_3$-$v_2$ and $v_4$-$v_2$ correlations have moderate $p_{\rm T}$ dependence in mid-central collisions. This might be an indication of possible viscous corrections for the equilibrium distribution at hadronic freeze-out.
The results presented in this article can be used to further optimize model parameters and put better constraints on the initial conditions and the transport properties of nuclear matter in ultra-relativistic heavy-ion collisions. However, the $p_{\rm{T}}$ dependences in $NSC(m,n)$ are not shown clearly.

We introduced the Scalar Product method (SP method) to measure $SC(m,n)$ and  $NSC(m,n)$, and checked with the results from QC method. Basically, most of the results are consistent in errors. However at some points, there are some deviations between the SP and QC methods. These differences are most pronounced in peripheral collision centrality regions. We investigate the reason of difference between two methods by testing and measuring $SC(m,n)$ with ToyMC, and PYTHIA jet implementation, but not able to fully explain the differences. 
At this point, we are not sure whether different methods respond differently to flow fluctuations or if we can rule out non-flow effects in the end. The answer might provide a hint for different sensitivities to flow fluctuations and non-flow effect. This might be a nice piece of material by itself in further study.

Even though, there are some missing parts on this analysis, such as no clear $p_{\rm{T}}$ dependence of  $NSC(m,n)$ and absence of explanation for different method response, it provides quantitative hints for the comprehensive understanding of hydrodynamical behavior of collide system evolution. Also new observables $SC(m,n)$ promise to provide additional constraints on the initial state phenomena and dynamical evolution and its fluctuation without event-by-event shape engineering. Since Run2 data from LHC is approaching with a new highest record of center-of-mass energy and much higher amount of events, more interesting analysis including correlations and fluctuations will be studied again.

