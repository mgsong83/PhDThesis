% !TEX root = main_org.tex


\chapter{Conclusion and Outlook}

As the strong evidence of QGP, flow has been studied in detail during the past few decades. But only a few studies have been done about correlation between flow harmonics and leads to the following questions; How do $v_n$ and $\epsilon_n$ fluctuate and what is the underlying probability density function ($p.d.f$) of their distribution? How are the initial geometry fluctuations reflected in differential flow measurements? What is the relationship between different harmonic event plane angle? What is the relationship between the flow coefficient of different harmonics? The answers to above questions (especially for the last) $Symmetric Cumulants$ ($SC(m,n)$) have been introduced as the first and only observable to measure correlation between flow harmonic ``magnitudes" $v_m$ and $v_n$.

As the result of $SC(m,n)$, we found that the correlation between $v_2$ and $v_4$ is positive(correlated), and the correlation between $v_2$ and $v_3$ is negative(anti-correlated). The various hydrodynamic calculations and model simulations were studied together as a reference. The large differences between data and MC-Glauber studies confirmed that the correlation were not able to explained by only linear response of initial conditions, and there is certain non-linear contribution of initial fluctuations. 

 None of the existing models and hydrodynamic calculation with the different parametrizations of the temperature dependence of $\eta/s$ couldn't exactly capture the data quantitatively. Furthermore, the sign of $v_3$-$v_2$ correlation in most central collision range(0-10\%) was found to be different between the data and hydrodynamic model calculations.  In the most central collisions the anisotropies originate mainly from fluctuations, i.e. the initial ellipsoidal geometry characteristic for mid-central collisions plays little role in this regime.  It is suggested that the $SC(m,n)$ is more sensitive to both initial conditions and hydrodynamic property $\eta/s$ than single flows.  

The analysis with higher order flow harmonics provides that the correlation between higher order flow harmonics and lower order harmonics is likely due to the non-linear contributions. It  also indicates that the higher order flow can be understood as the superposition of the lower order flow harmonics. This analysis will help constrain the theoretical description of the fluid close the freeze-out temperature which is probably the least understood part of hydrodynamic calculation. \cite{Teaney:2012ke} \cite{Yan:2015dh}

We introduce the Scalar Product method (SP method) to measure $SC(m,n)$ and  $NSC(m,n)$, and checked with the results from QC method. Basically, most of the results are consistent in errors. However at some points, there are some deviations between the SP and QC methods. These differences are most pronounced in peripheral collision centrality regions. We investigate the reason of difference between two methods by testing and measuring $SC(m,n)$ with ToyMC, and PYTHIA jet implementation, but not able to fully explain the differences. 

In the study about $p_T$ dependence, we found clear $p_T$ dependence in original $SC(m,n)$. But although there were many theoretical predictions(with Hydrodynamic and AMPT simulation), we were unable to find $p_T$ dependence in  $NSC(m,n)$ (with QC method). The recent study about $p_T$ dependence of flow and its direction point out that there are certain correlation between flow origin and the its $p_T$. Even though we see  both $p_T$ dependence of $SC(m,n)$ and  $NSC(m,n)$ with scalar product method we are not able to say that $p_T$ dependence are come the correlation itself or the other effects at this moment.

At this point, we are not sure whether different methods respond differently to flow fluctuations or if we can rule out non-flow effects in the end. The answer might provide a hint for different sensitivities to flow fluctuations and non-flow effect. This might be a nice piece of material by itself in further study.

Even though, there are some missing parts on this analysis, such as no clear $p_T$ dependence of  $NSC(m,n)$ and absence of explanation for different method response, it provides quantitative hints for the comprehensive understanding of hydrodynamical behavior of collide system evolution. Also  new observables $SC(m,n)$ promise to provide additional constraints on the initial state phenomena and dynamical evolution and its fluctuation without event-by-event shape engineering. Since Run2 data from LHC is approaching with a new highest record of center-of-mass energy and much higher amount of events, more interesting analysis including correlations and fluctuations will be studied again.

