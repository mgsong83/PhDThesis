
\clearpage 
\begin{abstract}

\begin{center}
{\LARGE \txtTitle}
\end{center}
\begin{flushright}
\parbox[t]{0.9\textwidth}
{\begin{flushright}
{\bf \txtAuthor}\\
\txtDepartment \\
The Graduate School, Yonsei University, Seoul, Korea
\end{flushright}}
\end{flushright}
\vspace{1em}

How did the universe begin? Relativistic heavy-ion collisions can answer this simple question since it can produce an extreme state of very hot and dense system similar to the state just after the Big Bang.

The existence of QGP at extreme conditions such as high temperature and energy density were proved by the Relativistic Heavy Ion Collider (RHIC) at BNL and Large Hadron Collider (LHC) at CERN. One of the most important probes to assess the properties of QGP is collectivity behavior of particle production in transverse direction. This phenomena were analyzed with Fourier's series transformations. Each order of Fourier harmonics is called ``flow"($v_n$). This flow provides not only evidence of existence of QGP matter, but also hints of the properties of created medium. 

The large $v_2$ (also known as ``elliptic flow") discovered at RHIC energies (and also found at LHC energy of 2.76TeV) were explained by pressure effect of  the almond-like shape of the collision overlap region. It also demonstrated that the QGP behaves like a strongly coupled liquid with a very small ratio of the share viscosity-to-entropy density ($\eta/s$). In this thesis the recent results of studies about flow and the few representative methods to measure flow are presented. 
 
 The other harmonics, such as odd and higher harmonics were explained as the result of fluctuation of initial geometry. However, this simple geometrical flow approach cannot explain the possible relation between two different flow harmonics. To measure and quantify the correlation between flow harmonics, new observable $Symmetric Cumulants$ have been introduced without biases originating from non-flow effects and any dependence on event planes. 
 
 The results from Pb + Pb collisions at ALICE with $\sqrt{s_{NN}}=2.76$TeV correlation between flow harmonics up to $5^{th}$ order are discussed, and the transverse momentum dependences of correlations are analyzed. Data from this study also are compared to model simulations from viscous hydrodynamics, AMPT, and HIJING models. 
 
 Together with existing measurements of individual flow harmonics, this analysis of heavy-ion collisions aims to better determine the initial conditons and $\eta/s$ as the transport properties of the system produced.

\vspace{\stretch{1}} \noindent
\hrulefill\\
{\bf Key words : }
\parbox[t]{0.8\textwidth}
{LHC, ALICE, Flow, Correlation, Fluctuation, elliptic flow($v_2$), $SC(m,n)$}

\end{abstract}
