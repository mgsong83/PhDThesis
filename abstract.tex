
\clearpage 
\begin{abstract}

\begin{center}
{\LARGE \txtTitle}
\end{center}
\begin{flushright}
\parbox[t]{0.9\textwidth}
{\begin{flushright}
{\bf \txtAuthor}\\
\txtDepartment \\
The Graduate School, Yonsei University, Seoul, Korea
\end{flushright}}
\end{flushright}
\vspace{1em}

How did the universe begin? Relativistic heavy-ion collisions can answer this question since it can produce an extreme state of very hot and dense system similar to the state just after the Big Bang.

The existence of QGP (Quark-Gluon-Plasma) at extreme conditions such as high temperature and energy density was proved by the Relativistic Heavy Ion Collider (RHIC) at BNL and Large Hadron Collider (LHC) at CERN. One of the most important probes to assess the properties of QGP is collective behavior of particle production in the plane transverse to the beam direction. These phenomena were analyzed with Fourier's series transformations. Each order of Fourier harmonic coefficient is called ``flow"($v_n$). This flow can provide not only evidence of existence of QGP, but also the properties of the QGP. 

The large $v_2$ (also known as ``elliptic flow") discovered at RHIC energies (and also found at LHC energy) were explained by pressure gradients of  the almond-like shape of the collision overlap region. It was also demonstrated that the QGP behaves like a strongly coupled liquid with a very small ratio of the share viscosity-to-entropy density ($\eta/s$). In this thesis the recent results of studies about flow and the few representative methods to measure flow are reviewed in Section 1.4. 
 
 The other harmonics, such as odd and higher harmonics were explained as the result of fluctuation of initial geometry. However, this simple geometrical flow approach cannot explain the possible relation between two different flow harmonics. To measure and quantify the correlation between flow harmonics, new multi particle observables, so called \textit{Symmetric Cumulants} have been introduced. These observables are particularly robust against few-particle non-flow correlations and have no dependence on event planes. 
 
 The results of \textit{Symmetric Cumulants} between different order flow harmonics in Pb-Pb collisions at $\sqrt{s_{NN}}=2.76$~TeV with ALICE detector at the Large Hadron Collider (LHC) and their transverse momentum dependence are presented. The results also are compared to model simulations from viscous hydrodynamics, AMPT, and HIJING models. 
 
 Together with existing measurements of individual flow harmonics, this analysis aims to better determine the initial conditons and $\eta/s$ as the transport properties of the system produced.

\vspace{\stretch{1}} \noindent
\hrulefill\\
{\bf Key words : }
\parbox[t]{0.8\textwidth}
{LHC, ALICE, Flow, Correlation, Fluctuation, elliptic flow($v_2$), $SC(m,n)$}

\end{abstract}
